\documentclass[a4paper,11pt]{article}

\usepackage{amsmath}
\usepackage{listings}
%package to fix images where they are defined with option H
\usepackage{float}

% Paquetes para matemáticas:
\usepackage{amsmath, amsthm, amssymb, amsfonts, amscd} % Teoremas, fuentes y símbolos.
\usepackage{tikz-cd} % para diagramas conmutativos
% for writing inference rules
\usepackage{proof}
% for quotes
\usepackage[autostyle]{csquotes}  
 % Nuevo estilo para definiciones
 \newtheoremstyle{definition-style} % Nombre del estilo
 {5pt}                % Espacio por encima
 {0pt}                % Espacio por debajo
 {}                   % Fuente del cuerpo
 {}                   % Identación: vacío= sin identación, \parindent = identación del parráfo
 {\bf}                % Fuente para la cabecera
 {.}                  % Puntuación tras la cabecera
 {\newline}               % Espacio tras la cabecera: { } = espacio usal entre palabras, \newline = nueva línea
 {}                   % Especificación de la cabecera (si se deja vaía implica 'normal')

 % Nuevo estilo para teoremas
 \newtheoremstyle{theorem-style} % Nombre del estilo
 {5pt}                % Espacio por encima
 {0pt}                % Espacio por debajo
 {\itshape}           % Fuente del cuerpo
 {}                   % Identación: vacío= sin identación, \parindent = identación del parráfo
 {\bf}                % Fuente para la cabecera
 {.}                  % Puntuación tras la cabecera
 {\newline}               % Espacio tras la cabecera: { } = espacio usal entre palabras, \newline = nueva línea
 {}                   % Especificación de la cabecera (si se deja vaía implica 'normal')

 % Nuevo estilo para ejemplos y ejercicios
 \newtheoremstyle{example-style} % Nombre del estilo
 {5pt}                % Espacio por encima
 {0pt}                % Espacio por debajo
 {}                   % Fuente del cuerpo
 {}                   % Identación: vacío= sin identación, \parindent = identación del parráfo
 {\scshape}                % Fuente para la cabecera
 {:}                  % Puntuación tras la cabecera
 {.5em}               % Espacio tras la cabecera: { } = espacio usal entre palabras, \newline = nueva línea
 {}                   % Especificación de la cabecera (si se deja vaía implica 'normal')

 % Teoremas:
 \theoremstyle{theorem-style}  % Otras posibilidades: plain (por defecto), definition, remark
 \newtheorem{theorem}{Theorem}[section]  
 \newtheorem{lemma}{Lemma}[section]
 \newtheorem{corollary}{Corollary}[section]  
 \newtheorem{exercise}{Exercise}[section]  

 % Definiciones, notas, conjeturas
 \theoremstyle{definition-style}
 \newtheorem{definition}{Definition}[section]
  
\usepackage[]{algorithm2e}  
\usepackage{algorithmic}
\usepackage{url}
\usepackage[french,british]{babel}
\lstdefinestyle{freefempp}{
  language=C++,
  basicstyle=\footnotesize\ttfamily,
  commentstyle=\itshape\color{violet},
  identifierstyle=\color{blue},
  morekeywords={ border, buildmesh, cos, dx, dy, fespace, fill, func,
    int2d, label, mesh, on, pi, plot, problem, sin, real, x, y},
  % float,
  frame=single,
  numbers=left
}
\lstset{style=freefempp}
\begin{document}
\section{Interaction paradigms}
The term Human-Computer Interaction (HCI) refers to the study of interaction between humans and computers. The scientific field of HCI has seen considerable advancements in the last 30 years, especially since the publication of Card, Moran and Newell’s Psychology of Human Computer Interaction in 1983. Most of the research efforts in HCI attempt to make use of human skills and abilities that are naturally developed by leading a life in this physical world. One can classify different paradigms of interaction that take place between humans and computers:

\subsection{Command lines}

Initial interfaces were command-line interfaces where a user sits in front of a terminal screen, and enters a line specific command to perform specific tasks and wait for a reply. Only a part of the terminal screen real estate was used. Some of the concepts that were introduced were the concept of infinite loop (prompt), the immediate execution of commands (enter), a syntax allowing obligatory or optional parameters, the maintenance of a history of commands and the introduction of keyboard short-cuts. They were further improved with the introduction of textual interfaces that made use of the entire terminal screen.

\subsection{WIMP (Windows, icons, menus and pointers)}

The WIMP paradigm relies on Windows, Icons, Menus and Pointers to interact with the user.  

\begin{itemize}
\item A \textbf{window} runs a self-contained program, isolated from other programs that (if in a multi-programmed operating system) run at the same time in other windows.
\item An \textbf{icon} acts as a shortcut to an action the computer performs (e.g., execute a program or task).
\item A \textbf{menu} is a text or icon-based selection system that selects and executes programs or tasks.
\item The \textbf{pointer} is an on-screen symbol that represents movement of a physical device that the user controls to select icons, data elements, etc.
\end{itemize}

Post-WIMP comprises work on user interfaces, mostly graphical user interfaces, which attempt to go beyond this paradigm.

The reason WIMP interfaces have become so prevalent is that they are very \textbf{good at abstracting work-spaces}, documents and their actions. Their analogous paradigm to documents as paper sheets or folders makes WIMP interfaces easy to introduce to other users. 

However, WIMP interfaces are \textbf{not optimal for working with complex tasks} such as showing 3D models (computer-aided design), working on large amounts of data simultaneously (interactive games) or simply portraying an interaction for which there is no defined standard widget. WIMPs are usually \textbf{pixel-hungry}, so given limited screen real estate they can distract attention from the task at hand.  

\subsection{Direct manipulation interface}

It was introduced in the context of the creation of office applications and the desktop metaphor. It involves continuous representation of objects providing actions that are \textbf{rapid, reversible, incremental and with feedback}. These actions should correspond to \textbf{manipulation of physical objects}. Example: resizing a graphical shape by dragging its corners or edges with a mouse or doing a drag-n-drop to delete a file.

Benefits: can make it easier for a user to learn and use an interface, having rapid, incremental feedback allows a user to make fewer errors and complete tasks in less time.

\subsection{Form filling}

It is preferred for bureaucratic and routine tasks. Introduces the notion of obligatory, optional or conditional fields, the notion of unfolding menus, the guide of the user during the process and the use of short-cuts. It allows the automatic verification of inputs. Example: online reservation of a flight. 

\subsection{Experiment with the four previous paradigms}

The experiment that can be found \url{http://cs211labs.epfl.ch} does a simulation of the purchase of five tickets in each of the four previous interaction modes. The conclusion is that the best paradigm depends on the task, the user and the details of the design. 

Depending of the degree of use of the tool we can have:

\begin{itemize}
\item Beginner: retains aspects of task of the real world without computer.
\item Intermediate: advanced features that translate the task from the real world to the computer.
\item Expert: aspects of computer program that don't have to do with the task itself.
\end{itemize}

Example: if we take OpenOffice Word, 

\begin{itemize}
\item Beginner: to write an essay one needs an index, an introduction and a body.
\item Intermediate: to structure the body of the essay one needs to separate the text in paragraphs with convenient parameters.
\item Expert: knows where one has to find the different options or how we call specific actions.
\end{itemize}

\subsection{Virtual reality}

Computer interactive simulation using sound and visual effects corresponding to real, imaginary or semi-imaginary environments.

Examples: Minecraft (characters represented by avatars although the world is not real), eye surgery learning.

\subsubsection{Immersive simulation}

An interactive computer simulation is \textbf{immersive} if the user receives similar stimulus (sight, smell, touch or sound) to the ones he would receive in the environment that is being simulated. 

Examples: a cave automatic virtual environment (CAVE) is an immersive virtual reality environment where video projectors are directed to between three and six of the walls of a room-sized cube. CAVE is typically a video theatre situated within a larger room. The user wears 3D glasses inside the CAVE to see 3D graphics generated by it. People using the CAVE can see objects apparently floating in the air, and can walk around them, getting a proper view of what they would look like in reality. A CAVE user's movements are tracked by the sensors typically attached to the 3D glasses and the video continually adjusts to retain the viewers perspective.

\subsubsection{Influence on the users}

A question remains to be answered. Do these virtual reality experiences really influence the users as the real world does?. Several experiments have been conducted in this sense, measuring physiological and cognitive responses. For instance, virtual  reality  users  hearts  have  pumped intensely  while  crossing  a  virtual pit (walking through a piece of wood surrounded by virtual cliffs). 

\subsubsection{Haptic devices}

Different \textbf{haptic} (related to the touch sense) \textbf{devices} has been implemented. They give the sensation of touch by applying forces (force feedback) or vibrations to the movements of users. This needs the presence of sensors for measuring the forces applied by the user. The user can then feel the specific resistance, elasticity or rugosity of the surface.

Applications:  surgery simulations, bass shakers vibrations in cinemas. 

A particularly interesting application is a \textbf{data glove or wired glove}. It is an input device for human-computer interaction worn like a glove. Various sensor technologies are used to capture physical data such as bending of fingers. Often a motion tracker, such as a magnetic tracking device or inertial tracking device, is attached to capture the global position/rotation data of the glove. 

These movements are then interpreted by the software that accompanies the glove, so any one movement can mean any number of things. Gestures can then be categorized into useful information, such as to recognize sign language or other symbolic functions.

Expensive high-end wired gloves can also provide haptic feedback, which is a simulation of the sense of touch. This allows a wired glove to also be used as an output device. Wired gloves are often used in virtual reality environments and to mimic human hand movement by robots.


\subsection{Augmented reality}

Augmented reality (AR) , is a live direct or indirect view of a physical, real-world environment whose elements are augmented by computer-generated sensory input such as sound, video, graphics or GPS data. Augmentation techniques are typically performed in \textbf{real-time}, that is, there are time limits that must be met. To this real setting one normally adds supplemental information like scores over a live video feed of a sporting event.

There are several hardware instruments that can be used to this goal. Using transparencies such as glasses or projections. We can even have interactive surfaces where we should ask ourselves whether is better to project from the upside or from the downside. A special characteristic of this surfaces is whether they are \textbf{multi-touch} or not. Multi-touch means that we can recognize the presence of more than one or more than two points of contact with the surface.

\subsection{Difference between virtual reality and augmented reality}

Virtual reality offers a digital recreation of a real life setting, while augmented reality delivers virtual elements as an overlay to the real world. 

Augmented reality is a type of virtual reality technology that blends what the user sees in their real surroundings with digital content generated by computer software. The additional software-generated images with the virtual scene typically enhance how the real surroundings look in some way. 

\subsection{Tangible interaction}

Tangible Interaction has come to be the 'umbrella term' used to describe a set of related research and design approaches which have emerged in several disciplines. It covers user interfaces and interaction approaches that emphasize:

\begin{itemize}
\item tangibility and materiality of the interface which is then connected with sensors with the computer. This sensors can be in the form of cameras, touch-tables or RFID.
\item the objects employed are specific to the task at hand which can be the visit to a museum to learn about an specific topic or directed to children to learn programming. It is very usual that this physical environments are thought to be multi-user to enhance teamwork. 
\end{itemize}

\subsection{Voice command devices}

These are interfaces that allow the user to \textbf{communicate using its voice} with the computer. A popular system for this is Siri. However, there is still a lot to do in human voice recognition to generalize these tools.  The level of precision depends on factors like the size of the vocabulary of the user, the environmental noise, the time that a particular user has been using it since Siri claims that it can learn with time or certain others semantic problems. There are also certain challenges in its usability.

\subsection{Gesture control}

This interface is based in the \textbf{recognition of human gestures}. In EPFL, reseracher Fr\'ed\'eric Kaplan has developed such a system to simulate tennis games. Microsoft Kinect is also an example. It enables users to control and interact with their console/computer without the need for a game controller, through a natural user interface using gestures and spoken commands.

\subsection{Brain–computer interface}

These interfaces allow the \textbf{communication of the user's brain activity to the computer}. In EPFL, researchers have obtained results controlling a wheel-chair. The thoughts of the user activate specific brain patterns that are recorded by electroencephalography (EEG) using a helmet with electrodes. These patterns are then interpreted by a computer that transmits a command to the chair. Interestingly, it takes some time to the user of these systems to adapt their thoughts to the machines they use. It is a process of mutual apprenticeship between human and machine.

\subsection{Bionic interfaces}

Bionics is the application of biological methods and systems found in nature to the study and design of engineering systems and modern technology. Therefore a bionic interface is an interface that profits of this knowledge to communicate the user and the computer.

An interesting experiment was carried out in EPFL by the team of Silvestro M\'icera to make a bionic hand with which patients could adjust their force to take objects and identify their shape and texture. The prosthesis contained some sensors capable of reacting to the tension of artificial tendons at translating them into electric impulses. This electric signals could then be transmitted to the actual external nerves of the patient arm.

\subsection{Wearables interfaces}

Wearable technology or wearables are electronic devices with microcontrollers that can be worn on the body as implants or accessories. The designs often incorporate practical functions and features.

This can be used for instance for \textbf{sousveillance}, which is the recording of an activity by a participant in it by means of wearing certain objects. It has also been used in the fashion industry. CuteCircuit was the first fashion company offering smart textile-based garments that create an emotional experience for their wearers using smart textiles and micro electronics.

\subsection{Interaction ambient}

The notion of ambient devices revolves around core concept of \textbf{immediate access to information}. The original developers of the idea (HYATT, ROSE, 2002) state that in the majority of cases an ambient device is designed to provide support to people in carrying out their everyday activities in an easy and natural way. An average person living in a modern society is being overloaded with abundance of information on a daily basis. Through the introduction of ambient devices into their day-to-day routine an individual gains an opportunity to decrease the amount of effort to process incoming data, thus rendering self more informed and productive (ROSE, 2002).

The key issue lies within \textbf{taking Internet-based content} (e.g. traffic congestion, weather condition, stock market quotes) \textbf{and mapping it into a single, usually one-dimensional spectrum} (e.g. angle, colour). According to one of the concept originators David L. Rose this way the data is represented to an end user seamlessly, and its procurement requires an insignificant amount of cognitive load.

One of the examples of the ambient device technology is Ambient Orb, introduced by Ambient Devices in 2002 (KIRSNER, 2002). The device itself was a glowing sphere which was continuously displaying data through perpetual changes in colour. Ambient Orb was customizable in terms of content and its subsequent visual representation. For instance, when the device was set to monitor a particular stock market index (e.g. NASDAQ), the Orb glowed green/red to represent the upward/downward movement of the stock prices; alternatively, it turned amber when the index is unchanged. Nabeel Hyatt stated that the device was marketed as an interior design item with additional functionality.

\subsection{Final thoughts}

Often, interfaces are said to be intuitive. However, we all have to learn how things work. The fact that we think that an interface is intuitive is more related with the fact hat \textbf{we forget that we have learnt it}. To properly asses the quality of a good interface one has to measure:

\begin{itemize}
\item the learning curve
\item the time needed to complete tasks
\item the number of errors
\item the user satisfaction
\end{itemize}

\subsection{Exercises}

\begin{exercise}
Suppose a website for reserving plane tickets. Classify the following knowledges into the different categories:
\end{exercise}
\begin{itemize}
\item If a user waits very long before confirming his choice the session is stopped and he has to restart all the process. (related to the computer translation)
\item A user has to enter the departure and arrival airports that are in the scope of the company's flights. (related to the computer translation)
\item Each airport of the world has a code name in three letters. For instance, GVA for Geneva. (related to the task)
\item In Swiss.com if the user wants to reserve a ticket in business class, he has to click on "more options" in the first dialogue window. (related to syntax)
\item I can order the possible flights in terms of the number of layovers or the price. (related to the computer translation)
\end{itemize}

\begin{exercise}
A software program for 3D rendering offers multiple functionalities to architects. What interaction paradigm is the most appropiate for the listed tasks. (It is possible to have several).
\end{exercise}
\begin{itemize}
\item Position a numeric image of the furniture and the equipment in the interior of the future kitchen. (direct manipulation)
\item position a numeric image of the future house on a real image of the ground to understand ints integration in the landscape. (augmented reality)
\item Make the the listing of the required windows for each room of the house. (form filling)
\item Propose to the clients an immersive visit in their future house. (virtual reality)
\end{itemize}


\section{Introduction to Computer Graphics}
\input{graphics/intro}
\section{Vision humaine}
\input{vision/vision}
\section{Cognition humaine}
\input{styles/cognition}
\section{Visualisation de l'information}
Il s'agit maintenant de d\'ecouvrir qu'est-ce qu'il y a dans mes donn\'ees. Comment montrer des choses apart de ce qui est gard\'e dans mon fichier. Par exemple dans le project de ce cours on montre le score obtenu. La visualisation des donn\'ees est tr\`es importante dans le monde de l'informatique d\'ecisionnelle (business intelligence) qui vise \`a donner les moyens pour collecter, consolider, mod\'eliser et restituer les donn\'ees d'une entreprise en vue d'offrir une aide \`a la d\'ecisiion et de permettre au d\'ecideur d'avoir une vue d'ensemble de l'activit\'e trait\'ee.

\subsection{Grammaire visuelles}

Ici on fait la visualisation en trois \'etapes:

1. \textbf{Placer les donn\'ees sur une image de base} de telle sorte que les propri\'t\'es visuelles de l'image refl\`etent les propri\'t'es abstraites des donn\'ees en particulier les relations entre donn\'ees. 

2. Cr\'eer une \textbf{grammaire visuelle} qui met en correspondance les variables des donn\'ees et les composantes graphiques. Si n\'ecessaire on cr\'ee une grammaire pour chaque dimension.

3. Mettre en \textbf{correspondance} un espace de n dimensions vers un espace de moindre dimensions par des m\'ethodes graphiques et des m\'ethodes statistiques.

\subsection{Principes de conception (Tufte)}

1. Choisir des unit\'es qui conservent du sens \`a travers les comparaisons. Exemple: wage minutes to by a hamburger in each country.

2. Choisir des unit\'es qui ont du sens pour le lecteur. Exemple: gross domestic product (GPD).

3. Choisir des intervalles pertinents. Exemple: a representation with big intervals can lead to not considering the whole information represented.

4. Evaluer les effets du choix des intervalles.

5. V\'erifier \textbf{l'int\'egrit\'e graphique}. Est-ce que l'importance visuelle correspond \`a la quantit\'e repr\'esent\'ee? Ici la perspective ou la r\'epresentation en 3D peuvent entra\^iner un pi\`ege.

6. Minimiser le "chart junk". Eviter les \'el\'ements qui n'apportent pas d'information et risquent de bruiter le message.

7. Optimiser le \textbf{"data ink ratio"}. Le data ink ratio c'est le quotient entre la quantit\'e de "ink" utilis\'ee to montrer le data entre la quantit\'e de "ink" utilis\'ee pour montrer le graphique. Elle devrait \^etre une magnitude \'elev\'ee outre on risque de utiliser trop d'\'elements graphiques pour r\'epresenter le data. Un donn\'ee peut s'identifier parce que si on l'efface, on perd de l'information.

8. Utiliser les \textbf{"small multiple"}. Il s'agit d'une s\'erie de graphiques similaires qui utilisent la m\^eme \'echelle et axes et qui permettent de les comparer facilement.

9. Montrer les co-variations. On montre sur un m\^eme graphique les variations des magnitudes r\'elationn\`ees . Ce qui permet \`a l'oeil de cr\'eer la correlation.

10. Montrer le contexte. Contextualiser l'information pour pr\'eciser son contenu.

\subsection{Distorsions g\'eom\'etriques}

Il n'existe pas de visualisation objective. Visualiser c'est communiquer.

On rencontre fr\'equentement des distorsions des visualizations qui peuvent \^etre forc\'ees ou n\'ecessaires.  

Exemples des \textbf{distorsions forc\'ees}: la repr\'esentation du globe terrestre sur le plan d'une carte. La Winker triple projection essaye de minimiser les distortion en termes de distance, angles et surfaces. La repr\'esentation d'une pyramide sur un plan, par exemple la r\'epresentation d'une montagne comme les Diablerets.

Il y a des autres distortions g\'eom\'etriques qui peuvent \^etre n\'ecessaires pour bien r\'epresenter l'objet sous \'etude. Par exemple le diagramme des lignes de m\'etro doit int\'egrer diff\'erent zones avec \textbf{diff\'erentes densit\'es}. Pour bien faire cette distorsion on peut utiliser des outils comme des \'echelles non-lin\'eaires, l'effet loupe ou des modes interactifs. 

On peut rencontrer des occassions ou il faut "tricher" pour rendre visible ce qui n'est pas visible. Par exemple si on repr\'sente les salaires et les ages d'un groupe des personnes et on veut savoir combien des personnes nous sommes en train d'\'etudier on peut ajouter du \textbf{bruit al\'eatoire ou "jitter"} aux donn\'ees pour rendre tous les points visible. En faisant \c{c}a, la visualization est fausse par respect \`a un axe mais plus correcte par rapport \`a la nature des donn\'es.

Une autre distortion n\'ecessaire peut \^etre \textbf{l'utilization des \'echelles incompatibles}. Par exemple si on veut representer une ascension alpiniste il est important de remarque l'ascension vertical m\^eme s'il est n\'egligeable en comparaison avec l'\'echelle horizontal. Ainsi on fait une exag\'eration dans certaines magnitudes. Une mesure pour d\'eterminer le degr\'e de distortion est le \textbf{Lie factor}. Il est le quotient entre l'effet montr\'e dans le graphique et l'effect montr\'e dans le data. 

Quelques distortions peuvent \textbf{communiquer des id\'ees}. Une carte standard peut \^etre distortion\'e pour montrer des informations sur la population ou le PIB de chaque pays. Le nombre des inscrits dans les cours de l'EPFL peut \^etre represent\'e par pays selon le nombre d'inscrits ou en r\'elation avec le nombre d'utilisateurs d'Internet.

\subsection{Erreurs fr\'equents dans la conception ou interpr\'etation des visualisations}

\begin{itemize}
\item Les couleurs ne fournissent pas d'information.
\item \textbf{L'empilement des graphiques} n\'ecessite de calculer mentalement les comparaisons.
\item Les \textbf{valeurs extr\^emes} \'ecrasent l’information. Une solution est l'utilisation d'\'echelles non-lin\'eaires par exemple l'\'echelle logarithmique.
\item Le syst\'eme g\'en\`ere automatiquement une \'echelle inappropri\'ee qui emp\^eche de faire des comparaisons correctes.
\item \textbf{Ordre des donn\'ees non justifi\'e}. Le pattern visuel produit d\'epend davantage de l’ordre des donn\'ees que des donn\'ees elles-m\^emes.
\item Ordre des donn\'ees innapropri\'e.
\item Les \textbf{points connect\'es} par des traits ne sont pas r\'eellement des donn\'ees li\'ees les unes aux autres.
\item \textbf{Sensibilit\'e de la moyenne aux valeurs extr\^emes}. C'est pour \c{c}a qu'on fait les analyses de variance. Dans les repr\'esentations graphiques de donn\'ees statistiques, la \textbf{bo\^ite \`a moustaches} est un moyen rapide de figurer le profil essentiel d'une s\'erie statistique quantitative.
\item \textbf{Sensibilit\'e des courbes de tendance}.
\item \textbf{Intepr\'eter une corr\'elation comme un lien de causalit\'e}. La corr\'elation peut \^etre en r\'ealit\'e caus\'ee par l'existence d'une \textbf{variable cach\'ee}. Ainsi probablement s'endormir avec une seule chaussure n'est pas la cause de se r\'eveiller avec un mal de t\^ete mais la cause peut \^etre tr\`es bien la consommation d'alcool.
\item Les \textbf{acronymes rares} pour l'utilisateur augmentent sa charge cognitive.
\item Le "split attention effect" augmente la charge cognitive.
\end{itemize}

The \textbf{split-attention effect} is a learning effect inherent within some poorly designed instructional materials. It is apparent when the same modality (e.g. visual) is used for various types of information within the same display. To learn from these materials, learners must split their attention between these materials to understand and use the materials provided.

Les \textbf{visualisations dynamiques} font usage de des utiles particuliers: changement d'\'echelle spatiale (zoom, croll,...), changement d'\'echelle temporelle, changement d'\'echelle des variables, rotations 2D ou 3D, changement de variables, "mouse over" (events that happen when the mouse is over the element)...

\subsection{Exercises}

\begin{figure}[H]
\centering
\makebox[\textwidth][c]{
\includegraphics[scale=0.35]{./images/exercise1.png}
}
\end{figure}

\begin{exercise}
En rempla\c{c}ant une visualisation en 3D par un treillis de vues 2D, quels principes de visualisation des donn\'ees sont-ils mis en oeuvre? 
\end{exercise}

Eviter le split attention effet (pas correcte)

Eviter l'occlusion de certaines donn\'ees (correcte)

Eviter le lie factor (pas correcte)

Utiliser les small multiples (correcte)

Utiliser les distorsions (pas correcte)















\section{M\'ecanismes des jeux}
\input{styles/jeux}
\section{Usability}
Usability is the degree to which a software can be used by specified consumers to achieve quantified objectives with effectiveness, efficiency, and satisfaction in a quantified context of use.

\subsection{Comment savoir si l'interface est bien con\c{c}ue?}

Parmi tous les principes deux sont sp\'ecialement important: les utilisateurs sont efficaces et on a suivi les \textbf{principes du design}.

\subsubsection{8 golden rules de Ben Schneiderman}
\begin{itemize}
\item \textbf{Strive for consistency}: Utiliser les m\^emes interactions et termes dans les m\^emes situations. C'est difficile si on d\'eveloppe seul, mais si on d\'eveloppe en \'equipe, c'est impossible. Il faut des guidelines strictes.
\item \textbf{Enable frequent users to use shortcuts} (addr\'ess\'e aux experts).
\item \textbf{Offer informative feedback}: tel que "erreur possible", "en cours de traitement". Ici, on inclu le micro-feedback tel que "action possible" accompagn\'e du changement de curseur et de couleur du bouton ou "action per\c{c}ue" accompagn\'e du changement de couleur du bouton et d'un son "click".
\item \textbf{Design dialog to yield closure} (final de l'action).
\item \textbf{Offer simple error handling}: expliquer l'erreur et comment le r\'eparer. Aussi, pr\'evenir l'erreur.
\item \textbf{Permit easy reversal of actions}: "back", "cancel", "undo", "undo history", "revert".
\item \textbf{Support internal locus of control}: we want to make sure the user feels in control of the software and confident in how to accomplish their tasks.  A user should never be wondering “How did I get to this screen?” or “What do I need to press to do my task?” Navigation and task activation should always be clear and well-marked.
\item \textbf{Reduce short-term memory load}: don’t make your user remember more things than necessary.  If your system has information scattered across different screens that are needed for one task, consolidate those screens.  If a user enters information into a form, don’t make them re-enter it during a validation sequence.
\end{itemize}

\subsubsection{10 usability heuristics de Jakob Nielsen}

\begin{itemize}
\item Visibility of system status: par exemple avec un status bar.
\item Match between system and the real world: par exemple, on doit voir si la structure du site web refl\`ete la structure de l'entreprise qu'il pr\'esente ou la structure des besoins des utilisateurs.
\item User control and freedom.
\item Consistency and standards. 
\item Error prevention. 
\item Recognition rather than recall (cognitive load) 
\item Flexibility and efficiency of use (shortcuts)
\item Aesthetic and minimalist design
\item Help users recognize, diagnose, and recover from errors
\item Help and documentation
\end{itemize}

\subsubsection{6 design principles de Don Norman}

\begin{itemize}
\item Consistency: D\'etecter et appliquer des patterns
\item Visibility: Toutes mes actions possibles sont visibles
\item \textbf{Affordance}: la forme de l'objet nous indique comment l'utiliser
\item \textbf{Mapping}: le lien entre l'action et son effet sont \'evidents
\item Feedback
\item \textbf{Constraints}:  Interfaces must be designed with restrictions so that the system can never enter into an invalid state. Constraints, or restrictions, prevent invalid data from being entered and prevent invalid actions from being performed.
\end{itemize}

\subsubsection{Deux probl\`emes}

1. Les principes sont partiellement contradictoires

Montrer toutes les options possibles.
Minimiser l'information \`a l'\'ecran.
Montrer toutes les options probablement int\'eressantes

2. Les principes ne sont pas des solutions.

Pour r\'ediger un bon r\'esum\'e, il faut qu'il...\\
soit bref mais capture les points importants\\
soit g\'en\'eral mais fournisse quelques d\'etails\\
soit objectif mais dot\'e d'un certain caract\`ere\\
r\'ev\`ele l'essentiel mais donne envie de lire\\
...

\subsubsection{La solution}

La solution passe par des principes de design qui inspirent des prototypes qui son test\'es par les utilisateurs, un proc\'ess qui peut boucler.

\subsubsection{Prototypage}

La phase de prototypage peut obtenir diff\'erents d\'egr\'es de fid\'elit\'e. Nous n'aurions pas la m\^eme fid\'elit\'e en papier qu'avec des outils informatiques. 

The Wizard of Oz method is a research experiment in which participants use a computer system that is actually being operated or partially operated by an unseen user. This unseen user is known as the “wizard”. The purpose of the “wizard” is to simulate the actions of the program in real time while watching the user in the other room through a video feed. Users are often unaware that the software is not operative. This technique is good for testing device concepts and techniques, however it requires more equipment and planning.

\subsubsection{Testing}

On peut mesurer diff\'erent aspects dans le "usability":

\begin{itemize}
\item Learnability: combien de temps un novice a-t-il besoin pour manipuler le logiciel (NO USER GUIDE)?
\item Efficiency: combien de temps est n\'ecessaire pour faire la t\^ache qu'il veut faire?
\item Memorability: si ils ne l'utilisent pas pendant un certain temps, est-ce que c'est difficile d'y revenir?
\item Errors: Combien d'erreurs? Pr\^etent-elles \`a cons\'equence?
\item Satisfaction: Est-ce agr\'eable d'utiliser l'application?
\end{itemize}

Les phases pour tester l'application sont:

\begin{itemize}
\item Apprendre l'application
\item Pr\'eparer une liste de t\^aches
\item Recruter 8 utilisateurs (\'echantillon stratif\'e) (1 jour)
\item Leur demander de faire la t\^ache en pensant \`a voix haute, prendre note voir et enregistrer (1  jour)
\item (Leur demander ce qu'ils pensent)
\item Analyser les r\'esultats
\item Convaincre les concepteurs (1 jour)
\end{itemize}

\begin{itemize}
\item Le testeur n'est pas le d\'eveloppeur
\item Le testeur est ind\'ependant du d\'eveloppeur
\item Le d\'eveloppeur n'est pas pr\'esent
\item Le d\'eveloppeur vous d\'eniera
\end{itemize}

\subsection{Comment savoir si une interface de type X est plus efficace que une interface de type Y?}

Nous allons concevoir une exp\'erience pour savoir si une interface tangible permet de mieux jouer qu'un joystick. La variable ind\'ependante est alors l'input device (joystick ou interface tangible) et les variables d\'ependantes peuvent \^etre le score, la vitesse, le nombre d'erreurs, l'apprentissage, le plaisir ou la long\'etivit\'e. Le but de l'exp\'erience c'est de mesurer si les variations la variable ind\'ependante produisent des variations dans la variable d\'ependante.

On peut d\'esigner une exp\'erience \textbf{"between subjects"} c'est \`a dire, on forme deux groupes des personnes pour tester chacune des modalit\'es de la variable ind\'ependente. Un groupe et le groupe c\^ontrole qui ne se voit pas affect\'e par la variable ind\'ependante \'etudi\'ee et l'autre est le group exp\'erimental qui se voit affect\'e par la variable ind\'ependante.

On peut d\'esigner une exp\'erience \textbf{"within subjects"} o\`u les personnes dans la premi\`ere condition passent ensuite dans la deuxi\`eme. Pour contre-balancer l'\textbf{effet d'ordre} on inverse l'ordre de passage des deux conditions pour la moiti\'e des sujets. Dans cette modalit\'e on a besoin alors de la moiti\'e des sujets et les sujets sont id\'entiques dans les deux conditions. Mais les effets de l'ordre sont souvent assez compliqu\'es \`a interpr\'eter. 

L'exp\'erience peut tester aussi des autres variables ind\'ependantes: le type de jeu, le niveau des joueurs, la r\'esolution spatiale du logiciel ou sa vitesse. 

Prennons par example l'input device et le niveau comme variable ind\'ependante, alors on aurait six groupes pour la m\'ethode "between subjects". Dans cette situation il est possible que l'effet de la pr\'emi\`ere variable ind\'ependante sur la variable d\'ependante affecte la valeur de la deuxi\`eme variable ind\'ependante. Si on a une variable continue et une variable discr\`ete, une manque de d\'ependance entre les variables est visible par le parall\'ellisme des graphes dans la variable discr\`ete. 

On peut m\^eme avoir une troisi\`eme dimension. Avoir plus de deux dimensions pose des probl\`emes. Ainsi, pour N facteurs et M modalit\'es par facteur, on devrait avoir \textbf{$M^N \cdot 20$} sujets. Les r\'esultats d'un tel exp\'erience sont aussi difficiles \`a interpr\'eter. Pour solutionner ces probl\`emes on peut d\'ecomposer en exp\'eriences successives \`a un ou deux dimensions pour mieux estimer la sensitivit\'e des variables.

Concevoir l'exp\'erience repose sur des principes tr\`es simples mais se heurte \`a moult subtils biais exp\'erimentaux:

\begin{itemize}
\item Est-ce que les sujets soumis aux diff\'erentes conditions \'etaient vraiment \'equivalents au d\'epart?

On peut consid\'erer ici l'\^age, niveau socio-culturel, scolaire, connaissances pr\'ealables, intelligence, raisonnement spacial ou motivation. Ces aspects peuvent \^etre m\'esur\'es par des questionnaires, dans le recrutement, avec des tests et avec la quantit\'e qu'on paye par exemple.

\item Est-ce que les sujets \'etaient inform\'es du but de l'exp\'erience? Est-ce que l'exp\'erimentateur \'etait vraiment neutre?

Un effet curieux c'est \textbf{l'effet Hawthorne}. Le simple fait de participer \`a une exp\'erience a une conse\'quence importante en termes de motivation. Ce qui peut affecter aux r\'esultat de la m\^eme. Un autre effet est \textbf{l'effet Pygmalion}. La performance des sujets est fonction des attentes de l'experimentateur. Un exemple est quand les proffesseurs ont plus d'expectatives dans certains \'el\`eves que dans des autres. Les \'el\`eves qui sont consid\'er\'es comme plus productifs vont produire plus que les autres. C'est pour \c{c}a que cet effet re\c{c}oit le nom de proph\'etie auto-r\'ealisatrice. L'exp\'erience de \textbf{conservation des liquides}, id\'ee de Jean Piaget, essaye d'identifier \`a quel \^age un enfant d\'eveloppe la capacit\'e de d\'etecter qu'une quantit\'e est pr\'serv\'ee m\^eme si on change le conteneur en forme ou taille. Cette exp\'erience a \'et\'e critiqu\'ee parce qu'elle souffre des effets mention\'es avant. En conclusion, il est difficile de faire des m\'ethodes en \textbf{"double-aveugle"} o\`u ni le sujet, ni l'exp\'erimentateur ne savent que le traitement est utilis\'e.

\item Est-ce qu'un \'el\'ement non-controll\'e peut expliquer les variations de la variable d\'ependante?

Dans toute exp\'erience il y a un nombre des variables interm\'ediaires ou variables de processus (comme la fatigue, la pr\'ecision, hauteur, log files, eye tracking, interactions) qui peuvent affecter au r\'esultat final. On appelle \textbf{effet de m\'ediation} \`a la relation entre variable ind\'ependante et variable d\'ependante qui s'explique statistiquement par une autre variable. Ainsi, une correlation entre les variables ne signifie pas causalit\'e mais la cause peut \^etre cach\'e par une autre variable.

Un autre probl\'eme est d'\'etablir si les variations qu'on observe dans les diff\'erents exp\'eriments sont vraiment \textbf{diff\'erences significatives} ou sont plut\^ot d\^u \`a une mal chance lors de la s\'election des sujets. Une m\'ethode pour ne pas obtenir une moyenne extr\^eme c'est de augmenter la taille de l'\'echantillon. Pour d\'eterminer si les diff\'erences sont significatives on compare alors la diff\'erence entre moyennes et la dispersion des \'echantillons.

Finalment, la technique de \textbf{meta-analyse}, combine les resultats de diff\'erents \'etudes scientifiques  pour avoir une plus grande s\^ur\'et\'e statistique.
\end{itemize}

\begin{exercise}
We compared new software for meetings the standard meeting rooms.

We selected 20 teams of 5 subjects and 20 teams of 10 subjects, i.e. 300 subjects. The ratio of women and men was the same in both conditions. The average age was respectively 34.3 and 33.9 years old.

Results : The time spent to agree on a solution was in average shorter with the new software. The degree of satisfaction was higher with the new software for teams of 5 but it was lower for the teams of 10. Teams
with the software produce in average shorter utterances and  participation was more balanced.

Find out which are:

\begin{itemize}
\item The 2 independent variables?
\item The 2 dependent variables?
\item The 2 controlled variables?
\item The 2 intermediate/process variables?
\item The 4 conditions?
\item The main effect?
\item The interaction effect?
\item Within subjects or between subjects
\end{itemize}
\end{exercise}

\begin{itemize}
\item The independent variables were the use or not of the software and the number of subjects in the group. 
\item The dependent variables were the time spent to agree to a solution and the degree of satisfaction.
\item The controlled variables were the number of women and men and the age of the subjects.
\item The intermediate variables were the number of utterances and the participation.
\item The four conditions were as follows: group of 5 not using software, group of 5 using software, group of 10 not using software and group of 10 using software.
\item The main effect was an improvement in the time spent to agree in a solution and not significant in the satisfaction (since we average the effects in the different independent variables).
\item The number of members in the team and the use or not of the software had an interaction effect: the degree of satisfaction was higher with the new software for teams of 5 but it was lower for the teams of 10.
\item The experience was carried out within subjects since every team experimented the use or not of the new software.
\end{itemize}

\begin{figure}[H]
\centering
\makebox[\textwidth][c]{
\includegraphics[scale=0.55]{./images/exercise4.png}
}
\end{figure}

































\section{Eye tracking}
\input{vision/eye}


\end{document}
